\documentclass{article}

% if you need to pass options to natbib, use, e.g.:
%     \PassOptionsToPackage{numbers, compress}{natbib}
% before loading neurips_2021

% ready for submission
% \usepackage{neurips_2021}

% to compile a preprint version, e.g., for submission to arXiv, add add the
% [preprint] option:
% \usepackage[preprint]{neurips_2021}

% to compile a camera-ready version, add the [final] option, e.g.:
%     \usepackage[final]{neurips_2021}

% to avoid loading the natbib package, add option nonatbib:
\usepackage[nonatbib,preprint]{neurips_2021}

\usepackage[utf8]{inputenc} % allow utf-8 input
\usepackage[T1]{fontenc}    % use 8-bit T1 fonts
\usepackage{hyperref}       % hyperlinks
\usepackage{url}            % simple URL typesetting
\usepackage{booktabs}       % professional-quality tables
\usepackage{amsfonts}       % blackboard math symbols
\usepackage{amsmath}        % advanced math environments
\usepackage{nicefrac}       % compact symbols for 1/2, etc.
\usepackage{microtype}      % microtypography
\usepackage{xcolor}         % colors
\usepackage{graphicx}
\usepackage[numbers]{natbib}
\usepackage{listings}
\usepackage[ruled,vlined]{algorithm2e}
\lstdefinestyle{pythonstyle}{
    language=Python,
    basicstyle=\ttfamily\footnotesize,
    keywordstyle=\color{blue},
    commentstyle=\color{gray},
    stringstyle=\color{green!50!black},
    numbers=left,
    numberstyle=\tiny\color{gray},
    stepnumber=1,
    numbersep=10pt,
    backgroundcolor=\color{yellow!10},
    frame=single,
    tabsize=4,
    captionpos=b,
    breaklines=true,
    breakatwhitespace=false,
    showspaces=false,
    showstringspaces=false,
    showtabs=false,
    morekeywords={self}  % Add keywords if needed
}
\lstset{style=pythonstyle}
\newcommand{\stream}[1]{\ensuremath{\mathcal{S}_{#1}}}
\newcommand{\streamty}[1]{\ensuremath{{#1}^{\omega}}}
\newcommand{\agent}{A}
\newcommand{\agentin}[1]{\ensuremath{\mathit{input}(#1)}}
\newcommand{\agentout}[1]{\ensuremath{\mathit{output}(#1)}}
\newcommand{\secref}[1]{Sec. \ref{sec:#1}}

\title{Final Paper: A Framework for Leveraging Small Language Models for Synthetic Data Generation for
Fine-Tuning}

% The \author macro works with any number of authors. There are two commands
% used to separate the names and addresses of multiple authors: \And and \AND.
%
% Using \And between authors leaves it to LaTeX to determine where to break the
% lines. Using \AND forces a line break at that point. So, if LaTeX puts 3 of 4
% authors names on the first line, and the last on the second line, try using
% \AND instead of \And before the third author name.

\author{%
  Zachary Grannan \\
  \And{}
  Owen Ren
}

\begin{document}

\maketitle

\begin{abstract}
Synthetic data is now commonly used in LLM post-training. However, there has
been less research into the use of domain-specific synthetic data for
fine-tuning purpose-built LLMs. In this work, we develop an agentic framework,
in the style of AgentInstruct \citep{mitra_agentinstruct_2024}, that facilitates
the generation of synthetic data for domain-specific LLM fine-tuning. Our
framework provides an API for defining agents and a declarative interface for
coordinating agents in data generation pipeline. Our framework facilitates
compositionality: for example, the caching of intermediate results in a pipeline
can be accomplished by adding a general-purpose caching agent rather than
modifying application code.  We evaluate our framework by using it to fine-tune
a model to answer questions about machine-learning papers. On a question-answer
task, our resulting fine-tuned model has performance comparable to an
alternative RAG-based approach. Our implementation, including the pipeline and scripts used
used in our evaluation is available at \url{https://github.com/zgrannan/532}. Our finetuned 
model and generated datasets are available at \url{https://huggingface.co/CPSC532}.
\end{abstract}

\section{Introduction}
Many commercial applications of large language models (LLMs) involve
domain-specific tasks requiring information not present in the LLM’s
training data. Retrieval-augmented generation (RAG)
\citep{lewis_retrieval-augmented_2020} is a popular technique for making such
information available. In RAG, the application injects data (e.g., from a
vector database) into the system prompt, with the hope that the LLM can use this
information to produce a response that incorporates relevant domain-specific
data. An advantage of the RAG-based approach is its flexibility and ease of use:
the application can use its own logic to determine how to enhance the prompts.
However, RAG systems have corresponding disadvantages. In particular,
incorporating a RAG architecture increases application and infrastructure
complexity.

Fine-tuning presents another approach. In the fine-tuning paradigm, the parameters of the LLM itself are modified by training the LLM on additional content. The fine-tuned LLM can be used as a "drop-in" replacement for a base language model: no changes to architecture or software are required.

Previous work \citep{balaguer_rag_2024,yang_fingpt_2023,wu_pmc-llama_2023} has
proposed methods for fine-tuning specialised LLMs, and techniques have been
developed to improve general reasoning abilities with synthetic data
\citep{shao_synthetic_2023,wang_self-instruct_2023}. However, there is
relatively less literature evaluating the performance of state-of-the-art
synthetic data generation techniques for fine-tuning domain-specific LLMs.
AgentInstruct \citep{mitra_agentinstruct_2024} provides a promising agent-based
approach, however the authors did not perform any evaluation for domain-specific
cases, and the AgentInstruct code is not readily available.

For this project, we intend to make the following contributions:

\begin{enumerate}
\item Develop an open-source agentic framework (in the style of AgentInstruct)
that can automatically generate synthetic data and fine-tune an LLM on that
data, and
\item Evaluate the performance of LLMs fine-tuned via our framework compared to
traditional RAG based techniques
\end{enumerate}

\section{Related Work}

With the rapid advancements of new large language models (LLMs), there is
concern that the amount of internet data that has traditionally been used to
train these models has been exhausted. Recent releases of the latest models
have used synthetic data during pre- and post-training \citep{abdin_phi-3_2024,
dubey_llama_2024, bai_qwen_2023}. Furthermore, the generative and reasoning
capabilities of LLMs have proven their effectiveness as tools for synthetic data
generation.

\subsection{Fine-Tuning with Synthetic Data}

Recent research into leveraging LLMs for synthetic data generation is split
into two approaches: seeded methods where LLMs are guided by an initial dataset
to generate new data and seedless methods where LLMs generate data without an
initial dataset. In Self-Instruct \citep{wang_self-instruct_2023}, the authors
leverage vanilla GPT-3 to generate instructions, input, and output samples to
fine-tune the same language model, demonstrating a 33\% improvement over the
original model. In Synthetic Prompting \cite{shao_synthetic_2023}, they leverage
an LLM to generate Chain-of-Thought prompts to enhance a language model’s
reasoning capabilities during inference. In TarGEN \citep{gupta2023targen}, the
authors present a seedless multi-step prompting strategy that generates
high-quality synthetic datasets using LLMs, including on domain-specific tasks
with minimal existing data. Recent approaches to using LLMs for synthetic data
generation have led to agentic approaches that generate large amounts of diverse
and high-quality data. In AgentInstruct \citep{mitra_agentinstruct_2024}, seed
data is leveraged to transform and generate new data using curated LLM agents
across 17 different skills ranging from reading comprehension, question
answering, creative writing to few-shot reasoning. The resulting fine-tuned
Orca-3 model trained on the 22 million instructions generated from AgentInstruct
outperforms GPT-3.5-turbo and GPT-4 in the MMLU and DROP benchmarks.

\subsection{Integrating Knowledge Retrieval and Fine-Tuning}

Recent literature in combining fine-tuning with RAG, RAFT \citep{zhang2024raft},
has shown improvements in performance across domain-specific datasets. REALM
\citep{guu_realm_2020} augments a language model with a neural knowledge
retrieval engine that is also trained during the pre-training and fine-tuning
stages.

\subsection{Efficient Fine-Tuning}

Advancements in both fine-tuning, in which an existing model’s weights are used
as the baseline for further training, and improvements in small language models
have drastically reduced the computation required to fine-tune a model. Methods
like LoRA \citep{hu2021lora} and QLoRA \cite{dettmers2024qlora} have further
reduced the computation requirements for LLM finetuning.

While Large Language Models have significantly advanced natural language
processing, small language models (SLM) have also seen improvements in text
generation, reasoning, and performance. Training techniques such as model
distillation use an LLM to fine-tune an SLM, transferring knowledge from the LLM
to the SLM. Furthermore, SLMs have advantages in cost, computation, and
flexibility over LLMs. While LLMs excel at generalisability, recent research has
shown that SLMs fine-tuned on domain-specific datasets can outperform
general-purpose LLMs for specific tasks.

\subsection{Domain-Specific LLMs}

Domain-specific LLMs, both those trained from scratch and those that fine-tune
an existing model, have been shown to outperform general-purpose LLMs. In the
former category, the closed-source BloombergGPT \citep{wu_bloomberggpt_2023},
trained from scratch on a dataset consisting of domain-specific and general
data, outperforms general-purpose LLMs on finance-related tasks.

Because training an LLM from scratch is extremely resource-intensive, approaches
based on fine-tuning are often more economically feasible. Such approaches
typically fine-tune an LLM by injecting additional knowledge and performing
instruction-tuning \citep{ouyang_training_2022}. FinGPT \citep{yang_fingpt_2023}
provides a framework for fine-tuning LLMs for finance-related tasks. The
lightweight model PMC-LLaMa \citep{wu_pmc-llama_2023} is developed by
fine-tuning a LLaMA model, and can outperform ChatGPT on various medical QA
tasks.

\section{Methodology}

\begin{figure}[h]
  \centering
  \includegraphics[width=1\textwidth]{methodology.jpg}
  \caption{An overview of our methodology. TODO}
\end{figure}

For our domain datasets that will be used as the seed data, we are still currently exploring datasets that can be used. Current ideas include research papers, company financial data, technical reports and manuals, or company specific FAQ data.
We will develop a multi agent synthetic data generation framework using small language models such as Llama-3.1-8B, Llama-3.2-3B, Qwen2.5-7B to synthetically generate high-quality diverse domain specific data.. Our immediate goal will be to create a framework for specializing in question answering tasks, similar to that of a traditional chatbot. Time permitting, we may expand this scope to also minimize hallucinations from model responses.
Next we will use the data to fine-tune LoRA adapters on the base model chosen. Since this will be an iterative process, we may also choose to perform a full supervised fine-tune of the model depending on resource limitations.

\subsection{Challenges}

One of the challenges we foresee may be overfitting on the fine-tuned model and the models inability to generalize to different writing styles and formats. We plan to mitigate this by increasing the diversity of our generated data. Another challenge is hallucination: previous research has shown that fine-tuning LLMs with new factual information increases their tendency to hallucinate (Gekhman et al. 2024). However, because we consider applications where the questions posed to the LLMs are related to data in the fine-tuning dataset, it's possible that this effect will be reduced (as opposed to questions unrelated to the domain, where a clear effect has already been observed in existing literature).

\section{Pipeline}\label{sec:pipeline}

The pipeline consists of three main components: the \textit{Question Generator}, the \textit{Answer Generator}, and the \textit{QA Refinement Generator}.

\begin{figure}[h]
  \centering
  \includegraphics[width=\textwidth]{methodology-overview.png}
  \caption{An overview of our synthetic data generation framework. Given a user-provided domain dataset, our
framework will employ small language models to create a synthetic dataset that can be used
to fine-tune a LLM for downstream Question Answering tasks.}
\end{figure}

\subsection{Data Preparation}
For the source dataset used in our framework, we used ArXiv papers published after 2024, specifically papers relevant to
small language models, fine-tuning LLM's and synthetic data generation.
This choice was motivated by several factors including:
\begin{itemize}
  \item Recent papers ensure that the content was not part of the pre-training data for the LLMs used in our framework
  \item The ease of access to the dataset, which is publicly available and well-structured
  \item Academic papers provide structured factual content that can be used to generate questions and answers and as a proxy for
  domain-specific data in other applications
\end{itemize}

To prepare the data into our framework, we chunked each paper into a fixed size and extracted the title to be used in the
question generator. 

\subsection{Question Generator}\label{sec:question-generator}

In the first step of the pipeline, the Question Generator takes the title of the paper along with a chunk of text from the paper
and generates an initial set of questions that can be answered using the text. To facilitate the generation of diverse questions,
we used a small language model to generate a mix of 'what', 'why', 'how', 'where', and 'summarize' questions, the full prompt can be found in the appendix?.

For each question in our framework, we ensure that the title information is included in the question to serve as context during fine-tuning. We hypothesize
that by including the title in the question, the model will be able to better retrieve the correct answer during downstream zero-shot question answering tasks on the fine-tuned
model. Intuitively, this serves as a form of conditioning for the language model during generation. See section ~\ref{sec:source-results} for more details.

To minimize generating duplicate or similar questions, we leverage a similarity check using the cosine similarity between the embeddings of the generated questions
and remove questions that have high similarity.

\subsection{Answer Generator}

Following the generation of questions, the Answer Generator takes the generated questions and the corresponding text chunk and generates answers for each of the questions.
In this step, we prioritize factual accuracy, organized structure of the response, and clarity. This step can be seen as a simple form of question answering, where the model is
tasked with generating the answer to a question given the context of the text chunk.

\subsection{QA Refinement Generator}
To refine each question-answer pair, we generate a new set of questions that are based on the original question and the answer pair
with the goal of generating questions that approaches the original question from a different angle. This step is critical in
generating more diverse qa pairs while also generating questions that are more challenging and involve more reasoning than the original question.

We generate an answer for each of the refined questions using a similar process as the Answer Generator but using the original answer
and smaller chunks retrieved from a similarity search as context. The extra step in retrieving smaller chunks of text is to ensure that the model
has context to answer the question that is not directly in the original text chunk. Intuitively, we take inspiration from Retrieval Augmented Generation
(RAG) frameworks using vector databases to generate answers in the QA Refinement Generator.

\subsection{Result}

Upon completion of the pipeline, we have a set of question-answer pairs that can be used to fine-tune a small language model for downstream QA tasks.
We perform simple heuristics to ensure that questions that could not be answered or may not exist in the original text are removed from the dataset.

\subsection{Algorithm}

\begin{algorithm}[H]
  \SetAlgoLined
  \KwData{PDF documents $D$, document chunk size $c_d$, retrieval chunk size $c_r$ $(c_d > c_r)$, overlap sizes $(o_d, o_r)$}
  \KwResult{Set of question-answer pairs $QA$}
  
  Initialize vector store $V$ and empty result set $QA$\;
  \ForEach{document $d \in D$}{
      Extract metadata (title, source) from $d$\;
      Chunk document into segments of size $c_d$ with overlap $o_d$\;
      
      \tcp{Embedding Pipeline for RAG}
      Create smaller retrieval chunks of size $c_r$ with overlap $o_r$ from document\;
      Embed retrieval chunks and store embeddings in vector store $V$\;
      
      \tcp{Question Generation Phase}
      \ForEach{document chunk $c$ of size $c_d$}{
          Generate base questions $Q_b$ using LLM\;
          Remove similar questions using embedding similarity\;
          Add source information to questions $Q_b$\;
          \ForEach{question $q \in Q_b$}{
              Generate answer $a$ using document chunk $c$ as context\;
              Add $(q,a)$ to base QA pairs\;
          }
      }
      
      \tcp{Question Refinement Phase}
      \ForEach{$(q,a)$ in base QA pairs}{
          Generate refined questions $Q_r$ based on $(q,a)$\;
          Remove similar questions using embedding similarity\;
          Add source information to questions $Q_r$\;
          \ForEach{question $q_r \in Q_r$}{
              Retrieve top $k$ most similar retrieval chunks $c_r$ from $V$\;
              Generate refined answer using retrieved chunks $c_r$ and initial answer $a$\ using LLM;
              Add refined QA pair to $QA$\;
          }
      }
  }
  \Return{$QA$}\;
  \caption{Synthetic QA Pair Generation Pipeline}
  \end{algorithm}

The algorithm above details our complete pipeline for generating synthetic question-answer pairs 
from domain-specific documents, incorporating both base question generation and refinement phases
with retrieval-augmented generation for improved quality and diversity of generated questions.


\subsection{Token Usage and Costs}

To estimate the total token usage and costs of running our synthetic data generation framework, 
we parameterize the token usage based on document size, chunk size, and prompt tokens for each compoennt. 

\section*{Parameter Definitions}

To estimate the total token usage for the pipeline, the following parameters are used:

\begin{itemize}
    \item \( N \): Total number of tokens in the document being processed.
    \item \( c_d \): Document chunk size used for question and answer generation.
    \item \( c_r \): Retrieval chunk size used during the question answer refinement. 
    \item \( k \): Number of top retrieval chunks selected during the retrieval process in question answer refinement. 
    \item \( n_q \): Number of base questions generated per document chunk. \( n_q = 10 \).
    \item \( n_r \): Number of refined questions generated during the question answer refinement phase. We empirically find that  \( n_r  \approx  5 \).

\end{itemize}

\subsection*{Prompt Token Sizes}
The prompt tokens represent fixed overheads for specific tasks in the pipeline:
\begin{itemize}
    \item \( T_{qg} \): Tokens required for the Question Generator prompt. \( T_{qg} = 500 \).
    \item \( T_{ag} \): Tokens required for the Answer Generator prompt. \( T_{ag} = 160 \).
    \item \( T_{qref} \): Tokens required for the Question Refinement prompt. \( T_{ref} = 100 \).
    \item \( T_{aref} \): Tokens required for the Answer Refinement prompt.  \( T_{ar} = 200 \).
    \item \( T_s \): Tokens required to add source information to questions.  \( T_s = 160 \)
    \item \( T_{qa}\): Tokens of generated question and answer pair. We empirically find that \( T_{qa} \approx 200 \).
\end{itemize}



\subsubsection*{Prompt Token Calculation}

To estimate the total token usage for the pipeline, we calculate the number of tokens required for each component of the pipeline
and sum them up to get the total token usage per document.
\begin{align*}
\text{Number of tokens for Question Generation} & = \left(T_{qg} + c_d\right) \cdot \frac{N}{c_d} \\
\text{Number of tokens for Answer Generation} & = \left(T_{ag} + c_d\right) \cdot \frac{N}{c_d} \cdot n_q \\
\text{Number of tokens for Question Refinement} & = \left(T_{qref} + T_{qa}\right) \cdot \frac{N}{c_d} \cdot n_q \\
\text{Number of tokens for Answer Refinement} & = \left(T_{aref} + T_{qa} + \left(k \cdot c_r\right)\right) \cdot \frac{N}{c_d} \cdot n_r \\
\text{Number of tokens for Source Addition} & = T_s \cdot \frac{N}{c_d} \cdot (n_q + n_r)
\end{align*}

\section{Implementation}


* Talk about the implementation details of the framework. 
* Unsloth, LMStudio 
* How we finetuned

\section{Evaluation}
We will evaluate our approach with respect to two research questions:

\begin{itemize}
\item RQ1: How does our approach compare to RAG-based approach w.r.t. QA tasks?
\item RQ2: How does our approach compare to RAG-based approach w.r.t. hallucinations?
\end{itemize}

To evaluate performance, we will compare responses on a curated test dataset
against a baseline retrieval-augmented generation system and evaluate the
responses using an oracle model: previous research has shown that LLMs are
effective at approximating human preferences related to chatbot output
\citep{zheng_judging_2023}. We will leverage existing tools such as Ragas
\citep{ragas} for evaluation.

To evaluate our approach, we will use off-the-shelf evaluation suites. For example, we could consider one of the following domain-specific QA benchmarks:

\begin{itemize}
\item TriviaQA \citep{joshi_triviaqa_2017} - A dataset consisting of 650k question-answer-evidence triples, sourced from trivia enthusiasts.
\item HotpotQA \citep{yang_hotpotqa_2018} - Contains 113K question-answer pairs from Wikipedia, alongside supporting evidence.
\item PubmedQA \citep{jin_pubmedqa_2019} - Nearly 300k QA instances derived from PubMed abstracts. Each instance takes the form of a research question with possible answers yes/no/maybe.
\end{itemize}

During our evaluation, we will fine-tune existing open-source models such as
Phi-3 \citep{abdin_phi-3_2024} and Llama 3 \citep{dubey_llama_2024}, etc. We
will perform our approach on different sizes (e.g., for the Phi-3 series, 3.8B,
7B, and 14B models are available), considering how the different models perform
for RAG and fine-tuning approaches. We will also consider different quantization
levels.

\section{Results}\label{sec:results}

\begin{itemize}
    \item Results from RAG system for judge
    \item Results from different model quantization methods
    \item Results from different LoRA training parameters (time permitting)
    \item Results from comparing with and without source information
\end{itemize}

\begin{table}[h]
\centering
\caption{Comparison of Fine-tuned Model vs. RAG-based Model}
\begin{tabular}{lrr}
\hline
Model & Wins & Losses \\
\hline
Finetuned - No Sources & 76 & 63 \\
\hline
\end{tabular}
\label{tab:finetuned-vs-base}
\end{table}


\section{Conclusion}

We expect that the performance of our approach, compared to traditional
RAG-based approach, will depend on the extent to which the structure of the
domain-specific data resembles the way it is used in an application. When the
structure is similar, we expect RAG-based approaches to exhibit comparable
performance, because the relevant data can be easily found using standard
retrieval metrics based on the distance between embeddings. However, when the
structure of the data is different, we expect our approach to yield better
results: the "pre-processing" of the data via agentic workflows and subsequent
fine-tuning on that data should hopefully allow the model to memorise the data
in a way such that it can use it effectively later.


\clearpage

\bibliographystyle{abbrvnat}
\bibliography{bib}

\clearpage
\appendix
\section{Appendix}

\subsection{Work Split}

\textbf{Zack:}
\begin{itemize}
    \item Framework API design
    \item Performance Comparison Evaluation
    \item LLM-As Judge Implementation
\end{itemize}

\vspace{1em}

\textbf{Owen:}
\begin{itemize}
    \item QA Pair Pipeline Logic
    \item Fine-tuning implementation
    \item Hallucination and Token Usage Evaluation
    \item RAG Implementation
\end{itemize}

\subsection{Additional Tables}

\begin{table}[h]
   \centering
   \caption{Papers, number of questions generated, and their corresponding page counts.}
   \label{tab:questions_and_pages}
   \begin{tabular}{p{8cm} r r}
   \toprule
   \textbf{Paper Name} & \textbf{Questions Generated} & \textbf{Pages} \\
   \midrule
   What is the Role of Small Models in the LLM Era: A Survey & 283 & 25 \\
   Does Fine-Tuning LLMs on New Knowledge Encourage Hallucinations? & 233 & 20 \\
   AgentInstruct Toward Generative Teaching With Agentic Flows & 218 & 32 \\
   Small Language Models: Survey, Measurements, and Insights & 215 & 22 \\
   LONGCITE: ENABLING LLMS TO GENERATE FINEGRAINED CITATIONS IN LONG-CONTEXT QA & 182 & 24 \\
   WHEN SCALING MEETS LLM FINETUNING: THE EFFECT OF DATA, MODEL, AND FINETUNING METHOD & 150 & 20 \\
   Finetuning LLMs for Enterprise: Practical Guidelines and Recommendations & 146 & 17 \\
   Orca-Math: Unlocking the potential of SLMs in Grade School Math & 113 & 14 \\
   From Artificial Needles to Real Haystacks: Improving Retrieval Capabilities in LLMs by Finetuning on Synthetic Data & 105 & 13 \\
   RAFT: Adapting Language Model to Domain Specific RAG & 104 & 12 \\
   \bottomrule
   \end{tabular}
\end{table}

\begin{table}[h]
    \centering
    \caption{Cosine Similarity Statistics for Model and Data Combinations Across Different LoRA rank R}
    \label{tab:cosine_similarity_context_sizes}
    \begin{tabular}{clccccc}
    \toprule
    \textbf{Test Set} & \textbf{R} & \textbf{Model} & \textbf{Mean} & \textbf{Std Dev} & \textbf{Min} & \textbf{Max} \\
    \midrule
    \centering Dataset 2 & 16  & Model 1 & 0.6983 & 0.1188 & 0.3872 & 0.9691 \\
    & 16  & Model 2 & 0.7670 & 0.1265 & 0.4052 & 0.9722 \\
    \addlinespace[0.5em]
    & 32  & Model 1 & 0.7710 & 0.1234 & 0.3907 & 0.9596 \\
    & 32  & Model 2 & 0.7120 & 0.1111 & 0.4141 & 0.9403 \\
    \addlinespace[0.5em]
    & 64  & Model 1 & 0.7016 & 0.1136 & 0.4067 & 0.9521 \\
    & 64  & Model 2 & 0.7077 & 0.1228 & 0.3803 & 0.9720 \\
    \addlinespace[0.5em]
    & 128 & Model 1 & 0.6949 & 0.1127 & 0.3900 & 0.9661 \\
    & 128 & Model 2 & 0.7200 & 0.1211 & 0.3914 & 0.9708 \\
    \midrule
    \centering Dataset 1 & 16  & Model 1 & 0.7669 & 0.1164 & 0.4498 & 0.9656 \\
    & 16  & Model 2 & 0.8120 & 0.1128 & 0.4776 & 0.9678 \\
    \addlinespace[0.5em]
    & 32  & Model 1 & 0.8201 & 0.1160 & 0.4627 & 0.9790 \\
    & 32  & Model 2 & 0.7633 & 0.1204 & 0.4601 & 0.9748 \\
    \addlinespace[0.5em]
    & 64  & Model 1 & 0.7643 & 0.1134 & 0.4982 & 0.9780 \\
    & 64  & Model 2 & 0.7681 & 0.1130 & 0.4603 & 0.9816 \\
    \addlinespace[0.5em]
    & 128 & Model 1 & 0.7662 & 0.1165 & 0.5043 & 0.9830 \\
    & 128 & Model 2 & 0.7747 & 0.1162 & 0.5186 & 0.9726 \\
    \bottomrule
    \end{tabular}
\end{table}

\begin{table}[t]
    \centering
    \caption{Unsloth LoRA Fine-Tuning Parameters for Llama 3.2 3B}
    \label{tab:lora-parameters}
    \begin{tabular}{l p{2cm} p{8.2cm}}
    \toprule
    \textbf{Parameter} & \textbf{Value} & \textbf{Description} \\
    \midrule
    \texttt{r} & 16, 32, 64, 128 & Rank of the low-rank matrices. Higher values retain more information but increase computational load. \\
    \addlinespace[3pt]
    \texttt{target\_modules} & q\_proj, k\_proj, v\_proj, o\_proj, gate\_proj, up\_proj, down\_proj &
    Layers targeted for LoRA adaptation. \\
    \addlinespace[3pt]
    \texttt{lora\_alpha} & 16 & Scaling factor for the LoRA updates. Higher values can speed up convergence but may risk instability. \\
    \addlinespace[3pt]
    \texttt{lora\_dropout} & 0 & Probability of zeroing out elements in LoRA layers during training for regularization. A value of 0 means no dropout. \\
    \addlinespace[3pt]
    \texttt{bias} & none & Determines how biases are handled in LoRA layers. Setting to ``none'' excludes biases, optimizing memory usage. \\
    \addlinespace[3pt]
    \texttt{use\_rslora} & True & Enables Rank-Stabilized LoRA, which adjusts the scaling factor to \texttt{lora\_alpha / sqrt(r)}, improving training stability. \\
    \bottomrule
    \end{tabular}
 \end{table}


\end{document}
