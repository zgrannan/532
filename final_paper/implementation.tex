\section{Implementation}\label{sec:implementation}


\subsection{Synthetic Data Generation Framework}

We chose Llama 3.1 8B as our LLM in our synthetic data generation framework, which is one the largest model that could fit on a single NVIDIA-3080. Initial experiments
were also attempted with Meta's recent Llama 3.2 3B model \cite{llama32} but was found to hallucinate more frequently than the 8B model and couldn't follow the instruction prompts as well.
To run inference locally, we deployed the models locally using LMStudio, an application that allows for easy deployment of large language models on local machines.

For question generation in our framework, we used JSON Mode to allow for easy extraction of the generated questions. While for answer generation, we used the default
mode to allow for diverse answers. Initial experiments were also attempted with JSON mode for both question and answer extraction but the question and answers
were found to be very basic and not diverse enough. This follows the study from \cite{tam2024letspeakfreelystudy} that found that structured generation and format constraints
generally lead to greater performance degradation in LLMs. We set temperature to 0 for all pipeline components to minimize hallucinations and randomness in the generation process.

The embedding models and the vector database that we used were nomic-embed-text-v1.5 and ChromaDB respectively.

\subsection{Dataset}

For our dataset, we picked 10 ArXiv papers published after 2024, specifically papers relevant to small language models, fine-tuning LLM's and synthetic data generation.
Using our framework, we generated a total of 1749 questions using a total of ~7.69M LLM tokens and averaging approximately 4.3k total tokens per question generated.
The breakdown of the number of questions generated and their corresponding page counts can be found in Table \ref{tab:questions_and_pages} and the token usage
for each pipeline component can be found in the appendix (Table A. \ref{tab:pipeline_tokens}). Estimating inference cost from cloud providers at \$0.20 per 1 million tokens,
the total cost of generating questions for the dataset would be approximately \$1.54. This further highlights the enormous cost savings that can be achieved
by leveraging small language models for data generation for question answering tasks.

\begin{table}[t]
   \centering
   \caption{Token Usage for each Pipeline Components}
   \label{tab:pipeline_tokens}
   \small
   \begin{tabular}{l r r r r}
   \toprule
   Pipeline Component & Mean & Mean & Mean & \# of \\
   & Prompt Tokens & Completion Tokens & Total Tokens & LLM Calls \\
   \midrule
   Question Generation & 2,988 & 456 & 3,444 & 79 \\
   Answer Generation & 4,453 & 327 & 4,780 & 351 \\
   Question Answer Refinement & 2,512 & 382 & 2,895 & 1790 \\
   Add Source to Question Agent & 215 & 52 & 268 & 2100 \\
   \bottomrule
   \end{tabular}
\end{table}
\subsection{Fine-tuning}


For the base model for fine-tuning, we chose Llama 3.2 3B model, which is smaller model than Llama 3.1 8B model used in the synthetic data generation framework and more
recently relased from Meta. This model also outperforms other small language models such as Gemma 2 2.6B and Phi 3.5-mini on various tasks.

We chose Unsloth for LoRA finetuning, which is an open-source framework designed to reduce memory usage and increase training speed for large language models such as Llama, Mistral,
Phi, and Qwen model families. The optimized framework, which is built upon custom Triton kernels for backpropagation steps, enables up to 70\% memory savings without decreases in
accuracy, which makes it suitable for fine-tuning large language models on consumer grade GPU's and environments such as Google Colab.

The parameters used for fine-tuning are shown in the appendix (Table A. \ref{tab:lora-parameters}). In our results, we evaluate the performance of the model when fine-tuned with various rank $\mathit{r}$ values.
The number of parameters for the LoRA adapters trained are shown in Table \ref{tab:rank-params}.


 \begin{table}[t]
    \centering
    \caption{LoRA Parameter Efficiency Analysis}
    \label{tab:rank-params}
    \begin{tabular}{l r r}
    \toprule
    \textbf{Rank (r)} & \textbf{Parameters} & \textbf{\shortstack{Trainable Ratio (\%)\\to base model}} \\
    \midrule
    16 & 24.3M & 0.76\% \\
    32 & 48.6M & 1.51\% \\
    64 & 97.0M & 3.02\% \\
    128 & 194.0M & 6.04\% \\
    \bottomrule
    \end{tabular}
\end{table}
