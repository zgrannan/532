\section{Related Work}

With the rapid advancements of new large language models (LLMs), there is
concern that the amount of internet data that has traditionally been used to
train these models have been exhausted. Recent releases of the latest models
have used synthetic data during pre- and post-training \citep{abdin_phi-3_2024,
dubey_llama_2024, bai_qwen_2023}. Furthermore, the generative and reasoning
capabilities of LLMs have proven their effectiveness as tools for synthetic data
generation.

\subsection{Fine-Tuning with Synthetic Data}

Recent research into leveraging LLMs for synthetic data generation are split
into two approaches: seeded method where LLMs are guided by an initial dataset
to generate new data and seedless methods where LLMs generate data without an
initial dataset. In Self-Instruct [Wang et al], the authors leverage vanilla
GPT3 to generate instructions, input, and output samples to finetune the same
language model , demonstrating a 33\% improvement over the original model. In
Shao et al’s paper on Synthetic prompting, they leverage a LLM to generate
Chain-of-Thought prompts to enhance a language model’s reasoning capabilities
during inference. In TarGEN, the authors present a seedless multi-step prompting
strategy that generates high-quality synthetic datasets using LLMs, including on
domain-specific tasks with minimal existing data. Recent approaches to using
LLMs for synthetic data generation have led to agentic approaches that generate
large amounts of diverse and high quality data. In AgentInstruct, seed data is
leveraged to transform and generate new data using curated LLM agents across 17
different skills ranging from reading comprehension, question answering,
creative writing to few shot reasoning. The resulting fine-tuned Orca-3 model
trained on the 22 million instructions generated from AgentInstruct outperforms
GPT-3.5-turbo and GPT-4 in the MMLU and DROP benchmarks.

\subsection{Integrating Knowledge Retrieval and Fine-Tuning}

Recent literature in combining fine tuning with RAG, RAFT \citep{zhang2024raft},
have shown improvements in performance across domain specific datasets. REALM
\citep{guu_realm_2020} augments a language model with a neural knowledge
retrieval engine that is also trained during the pre-training and fine-tuning
stages.

\subsection{Efficient Fine-Tuning}

Advancements in both fine-tuning, in which an existing model’s weights are used as the baseline for further training, and improvements in small language models has drastically reduced the computation required to fine-tune a model. Methods like LoRA and QLoRA have further reduced the computation requirements for LLM finetuning.

While Large Language Models have significantly advanced natural language processing, small language models (SLM) have also seen improvements in text generation, reasoning, and performance. Training techniques such as model distillation use a LLM is used to fine-tune a SLM in an effort to teach and train a SLM and knowledge from a LLM. Furthermore, SLMs have advantages in cost, computation, and flexibility over LLMs. While LLMs excel at generability, recent research has shown that SLMs fine-tuned on domain specific datasets can outperform general-purpose LLMs for specific tasks.

\subsection{Domain-Specific LLMs}

Domain-specific LLMs, both those trained from scratch and those that fine-tune
an existing model, have been shown to outperform general purpose LLMS. In the
former category, the closed-source BloombergGPT \citep{wu_bloomberggpt_2023},
trained from scratch on a dataset consisting of domain-specific and general
data, outperforms general-purpose LLMs on finance-related tasks.

Because training an LLM from scratch is extremely resource-intensive, approaches
based on fine-tuning are often more economically feasible. Such approaches
typically fine-tune an LLM by injecting additional knowledge and performing
instruction-tuning \citep{ouyang_training_2022}. FinGPT \citep{yang_fingpt_2023}
provides a framework for fine-tuning LLMs for finance-related tasks. The
lightweight model PMC-LLaMa \citep{wu_pmc-llama_2023} is developed by
fine-tuning a LLAMA model, and can outperform ChatGPT on various medical QA
tasks.
