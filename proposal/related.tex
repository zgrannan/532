\section{Related Work}

With the rapid advancements of new large language models (LLMs), there is
concern that the amount of internet data that has traditionally been used to
train these models has been exhausted. Recent releases of the latest models
have used synthetic data during pre- and post-training \citep{abdin_phi-3_2024,
dubey_llama_2024, bai_qwen_2023}. Furthermore, the generative and reasoning
capabilities of LLMs have proven their effectiveness as tools for synthetic data
generation.

\subsection{Fine-Tuning with Synthetic Data}

Recent research into leveraging LLMs for synthetic data generation is split
into two approaches: seeded methods where LLMs are guided by an initial dataset
to generate new data and seedless methods where LLMs generate data without an
initial dataset. In Self-Instruct \citep{wang_self-instruct_2023}, the authors
leverage vanilla GPT-3 to generate instructions, input, and output samples to
fine-tune the same language model, demonstrating a 33\% improvement over the
original model. In Synthetic Prompting \cite{shao_synthetic_2023}, they leverage
an LLM to generate Chain-of-Thought prompts to enhance a language model’s
reasoning capabilities during inference. In TarGEN \citep{gupta2023targen}, the
authors present a seedless multi-step prompting strategy that generates
high-quality synthetic datasets using LLMs, including on domain-specific tasks
with minimal existing data.
Recently, agentic approaches have emerged for synthetic data generation.
In AgentInstruct \citep{mitra_agentinstruct_2024}, seed
data is leveraged to transform and generate new data using curated LLM agents
across 17 different skills ranging from reading comprehension, question
answering, creative writing to few-shot reasoning. The resulting fine-tuned
Orca-3 model trained on the 22 million instructions generated from AgentInstruct
outperforms GPT-3.5-turbo and GPT-4 in the MMLU and DROP benchmarks.

\subsection{Integrating Knowledge Retrieval and Fine-Tuning}

Recent literature in combining fine-tuning with RAG, RAFT \citep{zhang2024raft},
has shown improvements in performance across domain-specific datasets. REALM
\citep{guu_realm_2020} augments a language model with a neural knowledge
retrieval engine that is also trained during the pre-training and fine-tuning
stages.

\subsection{Efficient Fine-Tuning and Small Language Models}

Advancements in both fine-tuning, in which an existing model’s weights are used
as the baseline for further training, and improvements in small language models
have drastically reduced the computation required to fine-tune a model. Methods
like LoRA \citep{hu2021lora} and QLoRA \cite{dettmers2024qlora} have further
reduced the computation requirements for LLM finetuning.

While Large Language Models have significantly advanced natural language
processing, small language models (SLM) have also seen improvements in text
generation, reasoning, and performance. \cite{chen2024rolesmallmodelsllm}\cite{lu2024smalllanguagemodelssurvey}.
Training techniques such as knowledge distillation \cite{xu2024surveyknowledgedistillationlarge} 
use an LLM to fine-tune an SLM, transferring knowledge from the LLM to the SLM. 
Furthermore, SLMs have advantages in cost, computation, and flexibility over LLMs. 
While LLMs excel at generalisability, recent research has shown that SLMs fine-tuned 
on domain-specific datasets can outperform general-purpose LLMs for specific tasks.

\subsection{Domain-Specific LLMs}

Domain-specific LLMs, both those trained from scratch and those that fine-tune
an existing model, have been shown to outperform general-purpose LLMs. In the
former category, the closed-source BloombergGPT \citep{wu_bloomberggpt_2023},
trained from scratch on a dataset consisting of domain-specific and general
data, outperforms general-purpose LLMs on finance-related tasks.

Because training an LLM from scratch is extremely resource-intensive, approaches
based on fine-tuning are often more economically feasible. Such approaches
typically fine-tune an LLM by injecting additional knowledge and performing
instruction-tuning \citep{ouyang_training_2022}. FinGPT \citep{yang_fingpt_2023}
provides a framework for fine-tuning LLMs for finance-related tasks. The
lightweight model PMC-LLaMa \citep{wu_pmc-llama_2023} is developed by
fine-tuning a LLaMA model, and can outperform ChatGPT on various medical QA
tasks.
